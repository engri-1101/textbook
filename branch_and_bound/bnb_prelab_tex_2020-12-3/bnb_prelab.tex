\documentclass[twoside]{article}%{combine}
%\usepackage{url}
\usepackage{../../tex/html}
\usepackage{epstopdf}
\usepackage{amsfonts,amsmath,color,amsthm,amssymb, enumerate, bbm, subfig}
\usepackage{graphicx}
%\usepackage[DIV=14,BCOR=2mm,headinclude=true,footinclude=false]{typearea}
%\usepackage[font=small,labelfont=bf]{caption}
\usepackage{hyperref}
\usepackage{tikz, etoolbox}

\usetikzlibrary{shapes}
\usetikzlibrary{arrows}
%\usepackage[margin=1in]{geometry}
\usepackage{graphicx,amsmath,gentium,tikz,caption}
\usetikzlibrary{patterns}
\usetikzlibrary{matrix,arrows,positioning,shapes}
\usetikzlibrary{arrows.meta}
\tikzset{
  a/.style={-{Stealth[scale=1.3,angle'=45]},semithick}
}
%\usepackage{xfrac,fontspec,unicode-math}
%\setmathfont[version=cambria]{Cambria Math}
%\mathversion{cambria}
\usepackage[letterpaper, portrait, margin=1.1in]{geometry}
\usepackage{amsmath,amsthm}
\usepackage{mathtools}
\newtheorem*{definition}{Definition}
\usepackage{tcolorbox}
\tcbset{colback=white,colframe=black}
\everymath{\displaystyle}

\makeatletter
\@ifundefined{namelength}{
\newlength{\namelength}
\settowidth{\namelength}{{\bf \Large Name: }}
\newlength{\namelinelength}
\setlength{\namelinelength}{\textwidth}
\addtolength{\namelinelength}{-\namelength}
}{}

\@ifundefined{vs}{
\newcommand*{\vs}[1]{\par
  \vspace*{#1\baselineskip}%
  \@afterindentfalse
  \@afterheading
}
}{}
\makeatother



\def\fancytitle#1#2#3{
      \centerline{\framebox{\framebox{ \parbox{.8\textwidth}{ \bf ENGRI 1101 \hfill
      Engineering Applications of OR \ \ \ \  Fall 2020 \hfill #3 #1 \\
\mbox{ }\hfill
      \hfill\mbox{ } \\[1mm] \mbox{ } \hfill{\Large \bf #2}\hfill
      \mbox{ }} }}}
      
\vs 2
}

\def\handout#1#2{\fancytitle{#1}{#2}{Handout}}
\def\review#1#2{\fancytitle{#1}{#2}{Review}}
\def\homework#1#2{\fancytitle{#1}{#2}{Homework}}
\def\exercises#1{\fancytitle{}{#1}{Exercises}}
\def\solution#1#2{\fancytitle{#1}{#2}{Solutions}}
\def\final#1#2{\fancytitle{#1}{#2}{Final}
      \noindent {\bf \Large Name:} \rule{\namelinelength}{0.5pt}
      \vspace*{\baselineskip}}
\def\prelim#1#2{\fancytitle{#1}{#2}{Prelim}
      \noindent {\bf \Large Name:} \rule{\namelinelength}{0.5pt}
      \vspace*{\baselineskip}}
\def\quiz#1#2{\fancytitle{#1}{#2}{Quiz}
      \noindent {\bf \Large Name:} \rule{\namelinelength}{0.5pt}
      \vspace*{\baselineskip}}
\def\lab#1#2{\fancytitle{#1}{#2}{Lab}
      \noindent {\bf \Large Name:} \rule{\namelinelength}{0.5pt}
      \vspace*{\baselineskip}}
\def\prelab#1#2{\fancytitle{#1}{#2}{Prelab}
      \noindent {\bf \Large Name:} \rule{\namelinelength}{0.5pt}
      \vspace*{\baselineskip}}

\raggedbottom

\begin{document}

\prelab{9}{IP, Branch-and-Bound, \& the Knapsack Problem}

\noindent
Objectives:

\begin{itemize}
\item   Introduce students to solving a small integer program by
branch-and-bound.
\item   Give students an appreciation of the difficulty of solving
integer programming problems exactly.
\end{itemize}

\noindent
Key Ideas:
\begin{itemize}
\item   linear programming relaxation
\item   branching
\item   branch-and-bound tree
\end{itemize}
\textbf{Reading Assignment:}
\begin{itemize}
\item
Handout 10 from the course-pack
\end{itemize}

\noindent
\textbf{Brief description:}
In this lab we will explore solving integer programs by branch-and-bound, including an example of the knapsack problem. We will also see how adding a cutting place helps in reducing the computation time and effort of the algorithm.

\smallskip
\noindent
\textbf{Prelab exercise:}
%\noindent
The knapsack problem is often
motivated by the following (not very realistic) story.
A hiker discovers a cave filled with valuable metal ingots.
Each ingot $i$ has a specified weight $w_i$, and a specified
value $v_i$. The hiker has the possibility of carrying out of
the cave a knapsack containing a subset of the ingots, provided
that the total weight is no greater than a specified weight $W$.
Suppose that we consider the integer programming formulation
of this problem: that is, maximize
$$\sum_{i=1}^n v_i x_i,$$
\noindent
subject to
$$\sum_{i=1}^n w_i x_i \le W,$$
$$0 \le x_i \le 1, {\mbox{ integer, }} i=1,\ldots,n,$$
where we are given the input data $W=13$ and
\begin{center}
\begin{tabular}{c|c|c}
$i$ & $v_i$ & $w_i$ \\
\hline
1 & 14 & 3\\
2 & 33 & 7 \\
3 & 11 & 2 \\
4 & 24 & 5 \\
5 & 5 & 1 \\
6 & 29 & 6 \\
7 & 20 & 4
\end{tabular}
\end{center}
\newpage
\noindent 
For this data, it turns out that if you solve the linear program where you ignore the constraint that each $x_i$ must be integer, you obtain the following optimal solution:
$$ x_1 = x_2 = x_4 = 0  \ \ \text{ and } x_3 = x_5 = x_6 =x_ 7 =1 .$$
What can you conclude about this input to the knapsack problem, i.e., the corresponding integer program?


\vspace{1in}
\noindent
Fortunately, your knapsack can hold more; $W=16$.
Now the optimal solution to the corresponding linear program is:
$$ x_1 = x_2 = 0  \ \ \text{ and } x_3 = x_5 = x_6 =x_ 7 =1 \text{ and } x_4 = .6 .$$
What can you conclude about the optimal value for this input to the knapsack problem?

\vspace{1in}

\noindent
Suppose that I tell you that, if you are allowed to select from
among just the first 6 items, then the optimal solution is take items 2,3,5 and 6,
which has total value 78. Suppose that I also tell you that if I change
$W$ to be 12, and then solve the knapsack problem again restricted to
taking items from among the first 6, the optimal solution is to take
pieces 1,3,5 and 6, which has total value 59. Use this information to deduce
the value of the original data for the knapsack problem (that is, selecting
from among all 7 items).
\end{document}