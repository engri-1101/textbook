\documentclass[twoside]{article}%{combine}
%\usepackage{url}
\usepackage{../../tex/html}
\usepackage{epstopdf}
\usepackage{amsfonts,amsmath,color,amsthm,amssymb, enumerate, bbm, subfig}
\usepackage{graphicx}
%\usepackage[DIV=14,BCOR=2mm,headinclude=true,footinclude=false]{typearea}
%\usepackage[font=small,labelfont=bf]{caption}
\usepackage{hyperref}
\usepackage{tikz, etoolbox}

\usetikzlibrary{shapes}
\usetikzlibrary{arrows}
%\usepackage[margin=1in]{geometry}
\usepackage{graphicx,amsmath,gentium,tikz,caption}
\usetikzlibrary{patterns}
\usetikzlibrary{matrix,arrows,positioning,shapes}
\usetikzlibrary{arrows.meta}
\tikzset{
  a/.style={-{Stealth[scale=1.3,angle'=45]},semithick}
}
%\usepackage{xfrac,fontspec,unicode-math}
%\setmathfont[version=cambria]{Cambria Math}
%\mathversion{cambria}
\usepackage[letterpaper, portrait, margin=1.1in]{geometry}
\usepackage{amsmath,amsthm}
\usepackage{mathtools}
\newtheorem*{definition}{Definition}
\usepackage{tcolorbox}
\tcbset{colback=white,colframe=black}
\everymath{\displaystyle}

\makeatletter
\@ifundefined{namelength}{
\newlength{\namelength}
\settowidth{\namelength}{{\bf \Large Name: }}
\newlength{\namelinelength}
\setlength{\namelinelength}{\textwidth}
\addtolength{\namelinelength}{-\namelength}
}{}

\@ifundefined{vs}{
\newcommand*{\vs}[1]{\par
  \vspace*{#1\baselineskip}%
  \@afterindentfalse
  \@afterheading
}
}{}
\makeatother



\def\fancytitle#1#2#3{
      \centerline{\framebox{\framebox{ \parbox{.8\textwidth}{ \bf ENGRI 1101 \hfill
      Engineering Applications of OR \ \ \ \  Fall 2020 \hfill #3 #1 \\
\mbox{ }\hfill
      \hfill\mbox{ } \\[1mm] \mbox{ } \hfill{\Large \bf #2}\hfill
      \mbox{ }} }}}
      
\vs 2
}

\def\handout#1#2{\fancytitle{#1}{#2}{Handout}}
\def\review#1#2{\fancytitle{#1}{#2}{Review}}
\def\homework#1#2{\fancytitle{#1}{#2}{Homework}}
\def\exercises#1{\fancytitle{}{#1}{Exercises}}
\def\solution#1#2{\fancytitle{#1}{#2}{Solutions}}
\def\final#1#2{\fancytitle{#1}{#2}{Final}
      \noindent {\bf \Large Name:} \rule{\namelinelength}{0.5pt}
      \vspace*{\baselineskip}}
\def\prelim#1#2{\fancytitle{#1}{#2}{Prelim}
      \noindent {\bf \Large Name:} \rule{\namelinelength}{0.5pt}
      \vspace*{\baselineskip}}
\def\quiz#1#2{\fancytitle{#1}{#2}{Quiz}
      \noindent {\bf \Large Name:} \rule{\namelinelength}{0.5pt}
      \vspace*{\baselineskip}}
\def\lab#1#2{\fancytitle{#1}{#2}{Lab}
      \noindent {\bf \Large Name:} \rule{\namelinelength}{0.5pt}
      \vspace*{\baselineskip}}
\def\prelab#1#2{\fancytitle{#1}{#2}{Prelab}
      \noindent {\bf \Large Name:} \rule{\namelinelength}{0.5pt}
      \vspace*{\baselineskip}}

\raggedbottom

\begin{document}

\prelab {8}{LP Formulations}

\noindent
\textbf{Objectives:}

\begin{itemize} 
\item Practice in formulating linear programming problems
\item Introduce the idea of linear relaxation
\item Demonstrate the relationship between an integer program and its linear relaxation
\end{itemize}

\noindent
\textbf{Brief description:} 
In this lab we will see an example of modeling a problem as a Linear Program (LP),
coding and solving it with ortools in a jupyter notebook, and start considering the question of what can be done when the desired
output of a mathematical model should be integer.

\noindent
\textbf{Pre-lab exercise}
Consider the following problem: (adapted from Hillier \& Lieberman) A new company is being set up to take
telephone orders. The phone lines are staffed around the clock. People are hired to work in 8-hour shifts,
starting at 6AM, 8AM, 12PM, 4PM, or 10PM. From survey data, the management has concluded that there
are minimum staffing requirements at certain times of the day, as given in the table below.

\begin{center}
\begin{tabular}{|l|c|}
\hline
Time period & Minimum Staff Size\\
\hline
6AM-8AM & 48 \\
\hline
8AM-10AM & 79 \\
\hline
10AM-Noon & 65 \\
\hline
Noon-2PM & 87 \\
\hline
2PM-4PM & 64 \\
\hline
4PM-6PM & 87 \\
\hline
6PM-8PM & 64 \\
\hline
8PM-10PM & 73 \\
\hline
10PM-Midnight & 82 \\
\hline
Midnight-6AM & 43 \\
\hline
\end{tabular}
\end{center}
A person is paid differently, depending on their assigned shift, where the daily salary is \$170, \$160, \$175,
\$180, \$195, for the five shifts, respectively. The management would like to decide the number of people to
be hired for each shift so as to minimize the total salary costs.

\begin{enumerate}
 
\item What do you think the decision variables should be?

\item How could you write the constraint that there are at least 79 people on staff from 8AM-10AM? Hint:
which workers will be on staff then?
\item
Your objective function is to minimize the cost incurred. 
How can you express this in terms of your decision
variables?

\end{enumerate}

\end{document}