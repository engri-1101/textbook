\documentclass[twoside]{article}%{combine}
%\usepackage{url}
\usepackage{../../tex/html}
\usepackage{epstopdf}
\usepackage{amsfonts,amsmath,color,amsthm,amssymb, enumerate, bbm, subfig}
\usepackage{graphicx}
%\usepackage[DIV=14,BCOR=2mm,headinclude=true,footinclude=false]{typearea}
%\usepackage[font=small,labelfont=bf]{caption}
\usepackage{hyperref}
\usepackage{tikz, etoolbox}

\usetikzlibrary{shapes}
\usetikzlibrary{arrows}
%\usepackage[margin=1in]{geometry}
\usepackage{graphicx,amsmath,gentium,tikz,caption}
\usetikzlibrary{patterns}
\usetikzlibrary{matrix,arrows,positioning,shapes}
\usetikzlibrary{arrows.meta}
\tikzset{
  a/.style={-{Stealth[scale=1.3,angle'=45]},semithick}
}
%\usepackage{xfrac,fontspec,unicode-math}
%\setmathfont[version=cambria]{Cambria Math}
%\mathversion{cambria}
\usepackage[letterpaper, portrait, margin=1.1in]{geometry}
\usepackage{amsmath,amsthm}
\usepackage{mathtools}
\newtheorem*{definition}{Definition}
\usepackage{tcolorbox}
\tcbset{colback=white,colframe=black}
\everymath{\displaystyle}

\makeatletter
\@ifundefined{namelength}{
\newlength{\namelength}
\settowidth{\namelength}{{\bf \Large Name: }}
\newlength{\namelinelength}
\setlength{\namelinelength}{\textwidth}
\addtolength{\namelinelength}{-\namelength}
}{}

\@ifundefined{vs}{
\newcommand*{\vs}[1]{\par
  \vspace*{#1\baselineskip}%
  \@afterindentfalse
  \@afterheading
}
}{}
\makeatother



\def\fancytitle#1#2#3{
      \centerline{\framebox{\framebox{ \parbox{.8\textwidth}{ \bf ENGRI 1101 \hfill
      Engineering Applications of OR \ \ \ \  Fall 2020 \hfill #3 #1 \\
\mbox{ }\hfill
      \hfill\mbox{ } \\[1mm] \mbox{ } \hfill{\Large \bf #2}\hfill
      \mbox{ }} }}}
      
\vs 2
}

\def\handout#1#2{\fancytitle{#1}{#2}{Handout}}
\def\review#1#2{\fancytitle{#1}{#2}{Review}}
\def\homework#1#2{\fancytitle{#1}{#2}{Homework}}
\def\exercises#1{\fancytitle{}{#1}{Exercises}}
\def\solution#1#2{\fancytitle{#1}{#2}{Solutions}}
\def\final#1#2{\fancytitle{#1}{#2}{Final}
      \noindent {\bf \Large Name:} \rule{\namelinelength}{0.5pt}
      \vspace*{\baselineskip}}
\def\prelim#1#2{\fancytitle{#1}{#2}{Prelim}
      \noindent {\bf \Large Name:} \rule{\namelinelength}{0.5pt}
      \vspace*{\baselineskip}}
\def\quiz#1#2{\fancytitle{#1}{#2}{Quiz}
      \noindent {\bf \Large Name:} \rule{\namelinelength}{0.5pt}
      \vspace*{\baselineskip}}
\def\lab#1#2{\fancytitle{#1}{#2}{Lab}
      \noindent {\bf \Large Name:} \rule{\namelinelength}{0.5pt}
      \vspace*{\baselineskip}}
\def\prelab#1#2{\fancytitle{#1}{#2}{Prelab}
      \noindent {\bf \Large Name:} \rule{\namelinelength}{0.5pt}
      \vspace*{\baselineskip}}

\raggedbottom

\begin{document}

\prelab {7}{LPs and the Simplex Method}

\noindent
\textbf{Objectives:}

\begin{itemize}
\item Introduce the simplex method to solve linear programs.
\item Understand the geometry of linear programs and the trajectory traced by the simplex method.
\end{itemize}

\noindent
\textbf{Key Ideas:}

\begin{minipage}[t]{.45\linewidth}
  \begin{itemize}
  \item dictionary
  \item feasible region
  \item isoprofit lines
  \end{itemize}  
\end{minipage}
\hfill

\vspace{1em}

\noindent
\textbf{Reading Assignment:}
\begin{itemize}
\item
Read Handout 8 on linear programming and the simplex method.
\end{itemize}

\noindent
\textbf{Prelab Exercise:}


\noindent
Open the tool at
\noindent
\url{
http://engri1101.orie.cornell.edu/2d_lp_example/2d_lp_example.html}.

\smallskip
\noindent
Use the objective function slider to review the optimal solution obtained by the graphical method.
What is the optimal value? (And then return the slider back to 0.)
\bigskip

\smallskip
\noindent
Use the iteration slider to advance step by step through the dictionaries computed by the simplex method. For each dictionary computed along the way, write out the objective function and the corresponding solution to that dictionary. 

\bigskip
\smallskip
Return to the dictionary for iteration 0. In the first iteration, the variable increased was $x_2$. Just by looking at the graph, determine the solution that would correspond to the dictionary found after the first iteration if, instead, we increased $x_1$.

\end{document}
